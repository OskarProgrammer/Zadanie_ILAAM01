% Creating a simple Title Page in Beamer
\documentclass{beamer}

% zawieranie znakow
\usepackage[T1]{fontenc}

% wysrodkowywanie tekstu
\usepackage{ragged2e}

% podswietlanie skladni
\usepackage[cache=false]{minted}

% tworzenie przerwy
\usepackage{setspace}\setstretch{1}

% wybor tla
\usetheme{Boadilla}

% szczegoly glownej strony
% tytuł
\title{Programowanie obiektowe w Pythonie}
% podtytuł
\subtitle{Czym jest, jak tworzyć klasy, jak ich używać?}
% autor
\author{Oskar}
%data
\date{\today}

% logo na każdym slajdzie z szerokością 1 cm
\logo{\includegraphics[width=1cm]{python-logo.png}}

% tworzenie bloku definicji
\newtheorem{twierdzenieBlok}{Definicja}

% rozpoczecie dokumentu
\begin{document}

% slajd glownej strony
\begin{frame}
    \titlepage
\end{frame}

% spis tresci
\begin{frame}{Spis Treści}
	\begin{Center}
		\tableofcontents[pausesections,hideallsubsections]
	\end{Center}
\end{frame}

% slajd czym jest programowanie obiektowe 
\section{Czym jest programowanie obiektowe}
\begin{frame}{Czym jest programowanie obiektowe?}
	    	\begin{twierdzenieBlok}
		    	\begin{Center}
			\footnotesize Programowanie obiektowe to metoda budowania programów komputerowych za pomocą obiektów. Obiekt łączy ze sobą grupę danych – często o różnych typach – i operujący na nich zestaw funkcji. Inaczej można powiedzieć, że obiekt zawiera dane składowe (pola) oraz funkcje składowe (metody) połączone w spójny element programu.
			\end{Center}
		\end{twierdzenieBlok}
\end{frame}

% slajd jak stworzyć klasę w języku python
\section{Jak stworzyć podstawowa klasę?}
\begin{frame}[fragile]{Jak stworzyć podstawową klasę?}
	\begin{twierdzenieBlok}
		\footnotesize Definiowanie klasy z konstruktorem:
  		\scriptsize
		\begin{minted}{python}
class NazwaKlasy(object):
	def __init__(self):
		pass
		\end{minted}
	\end{twierdzenieBlok}
\end{frame}


% slajd jak stworzyć metode i pole w klasie
\section{Jak stworzyć metodę i polę w klasie?}
\begin{frame}[fragile]{Jak stworzyć metodę i polę w klasie ?}
	\begin{twierdzenieBlok}
		\footnotesize Definiowanie metody:
  		\scriptsize
		\begin{minted}{python}
class NazwaKlasy(object):
	def nazwaMetody(self):#nasza metoda
		pass	
		\end{minted}
	\end{twierdzenieBlok}

	\begin{twierdzenieBlok}
		\footnotesize Definiowanie pola:
  		\scriptsize
		\begin{minted}{python}
class NazwaKlasy(object):
	nazwaPola = 0 #domyślna wartość	
		\end{minted}
	\end{twierdzenieBlok}
\end{frame}

% slajd jak użyć metody i pola
\section{Jak utworzyć instancję klasy i wywołać na niej metodę lub dostać się do pola?}
\begin{frame}[fragile]{Jak utworzyć instancję klasy i wywołać na niej metodę lub dostać się do pola ?}

			\scriptsize
			\tiny Aby utworzyć obiekt klasy, musimy najpierw ją zdefiniować z przykładowym polem, jak i metodą: 
			\begin{minted}{python}
class NazwaKlasy(object):
	przykladowePole = 0
	
	def przykladowaMetoda(self):
		print(f"Hello world!")

			\end{minted}
			
			\rule{\textwidth}{1pt}

			Aby utworzyć obiekt klasy musimy wywołać następującą instrukcję:
			\begin{minted}{python}

mojObiektKlasy = NazwaKlasy()

			\end{minted}

			\rule{\textwidth}{1pt}

			Aby wywołać metodę na obiekcie klasy:
			\begin{minted}{python}

mojObiektKlasy.przykladowaMetoda()

			\end{minted}

			\rule{\textwidth}{1pt}

			Aby dostać się do pola klasy:
			\begin{minted}{python}

print(mojObiektKlasy.przykladowePole)

			\end{minted}
	
\end{frame}

% slajd kończący 
\begin{frame}
	\begin{Center}
		\begin{spacing}{1} Prezentację przygotował Oskar !\end{spacing}
		\begin{spacing}{2} Dziękuję za uwagę !\end{spacing}
		\begin{spacing}{2} \includegraphics[width=1cm,height=1cm]{usmiechnieta_buzka.png} \end{spacing}
	\end{Center}
\end{frame}

% slajd bibliografia załącznikowej 
\section{Bibliografia załącznikowa}
\begin{frame}{Bibliografia załącznikowa}
\end{frame}

\end{document}
